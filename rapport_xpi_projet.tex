%% Based on a TeXnicCenter-Template by Tino Weinkauf.
%%%%%%%%%%%%%%%%%%%%%%%%%%%%%%%%%%%%%%%%%%%%%%%%%%%%%%%%%%%%%

%%%%%%%%%%%%%%%%%%%%%%%%%%%%%%%%%%%%%%%%%%%%%%%%%%%%%%%%%%%%%
%% HEADER
%%%%%%%%%%%%%%%%%%%%%%%%%%%%%%%%%%%%%%%%%%%%%%%%%%%%%%%%%%%%%
\documentclass[a4paper,10pt]{report}

\usepackage[utf8]{inputenc}
\usepackage[T1]{fontenc}
%% Language %%%%%%%%%%%%%%%%%%%%%%%%%%%%%%%%%%%%%%%%%%%%%%%%%
\usepackage[francais]{babel} %francais, polish, spanish, ...
\usepackage{lmodern} %Type1-font for non-english texts and characters

\usepackage{hyperref}

%% Packages for Graphics & Figures %%%%%%%%%%%%%%%%%%%%%%%%%%
\usepackage{graphicx} %%For loading graphic files


%%%%%%%%%%%%%%%%%%%%%%%%%%%%%%%%%%%%%%%%%%%%%%%%%%%%%%%%%%%%%
%% DOCUMENT
%%%%%%%%%%%%%%%%%%%%%%%%%%%%%%%%%%%%%%%%%%%%%%%%%%%%%%%%%%%%%
\begin{document}

\pagestyle{plain} %No headings for the first pages.


%% Title Page %%%%%%%%%%%%%%%%%%%%%%%%%%%%%%%%%%%%%%%%%%%%%%%
%% ==> Write your text here or include other files.
\pagenumbering{arabic}
%% The simple version:
\title{Projet d'XPI}
\author{Sébastien Bazin
\and Mayas Haddad
\and Ricky Keophila
\and Mohand-Saidh Hamitouche}
\maketitle
\tableofcontents

\chapter{Introduction}
Le présent document constitue le rapport accompagnant le rendu du projet d'XPI pour la promotion MIAGE de l'année 2013/2014.

\section{Problématique}
La problématique traitée par l'équipe de projet est celle de réaliser une plate-forme web d'achat et de location de films.
Voir le \href{https://www.lri.fr/~kn/files/projet_xpi_2013.pdf}{cahier des charges} pour plus d'informations.

\chapter{Choix de conception}

\section{Design Pattern MVC}

Pour conception de notre projet nous avons utilisé un Design Pattern nommée MVC (Modèle Vue Contrôleur). Ce modèle permet d'organiser et structurer l'interface graphique du projet.
Il est composé de trois entités distinct qui joue chacun un rôle dans l'interface.
Ces trois entités sont le Modèle,la Vue et le Contrôleur.

Le Modèle contient les données manipulées par le programme. Il assure la gestion de ces données et garantit leur intégrité. Il a pour but de gérer l'organisation des données. 
Le modèle offre des méthodes pour mettre à jour ces données (insertion suppression, changement de valeur). Il offre aussi des méthodes pour récupérer ses données. Dans le cas de données importantes, le modèle peut autoriser plusieurs vues partielles des données. 

La Vue permet d'afficher l'interface graphique pour l'utilisateur dans l'application.
La vue affiche les données qu'elle a récupérées auprès du modèle. Elle a aussi pour rôle de recevoir tous les actions de l'utilisateur (clic de souris, sélection d'une entrées, boutons, …) qui sont sont envoyés ensuite au contrôleur.
La vue peut aussi donner plusieurs vues, partielles ou non, des mêmes données et aussi la possibilité à l'utilisateur de changer de vue.


Le Contrôleur est chargé de la synchronisation du modèle et de la vue. Il reçoit tous les événements de l'utilisateur et enclenche les actions à effectuer. Si une action a besoin d'un changement des données, le contrôleur demande la modification des données au modèle et ensuite avertit la vue que les données ont changé pour que celle-ci se mette à jour. Certains événements de l'utilisateur ne concerne pas les données mais la vue. Dans ce cas, le contrôleur demande à la vue de se modifier.



\chapter{Choix d'implémentation}

\section{Twig pour les templates }

Twig est un moteur de template qui permet de séparer le code PHP du code HTML.

\section{PHP Orienté Objet pour la logique et le modèle}

Le PHP Orienté Objet permet d'avoir un code plus structuré, plus lisible, facilement réutilisable,maintenable, simple à utiliser. 

\section{MySQL }

parler de l'aspect que cela apporte niveaux sécurité.

\section{XML et XPath}

\chapter{Description des scénarios}

\section{Scénarios client}

Le scénario du client a été défini dans le cahier des charges et détaillé ci-dessous.
Un client doit pouvoir s’inscrire sur le site en renseignant ses informations personnelles. Une fois inscrit, il peut s’y connecter avec un son identifiant et parcourir le catalogue du site. Dans ce catalogue, il peut choisir un certain nombre de vidéos. Une vidéos est soit visionable pendant 48h en streaming soit téléchargeable sous la forme d’un fichier. Chaque location ou achat décrémente un certain montant sur le compte du client qui peut le recharger à tout moment depuis son interface de gestion. De plus, le site propose aussi d’autres produits comme des bons pré-payés qui permettent d’offrir des vidéos à des personnes tierces.

En revanche, nous n'avons pas pu mettre en place la partie où une vidéos est soit visionable pendant 48h en streaming soit téléchargeable sous la forme d’un fichier.

\section{Scénario admin}

Le scénario du admin a été défini dans le cahier des charges et détaillé ci-dessous.
L’administrateur a un contrôle total sur le site. Il peut en particulier :
— supprimer/ajouter des clients ou modifier des données sur leur compte 
— rajouter de nouvelles entrées au catalogue

\chapter{Répartition du travail}

\section{}

\subsection{}

\subsubsection{}

\paragraph{}

\subparagraph{}

\section{}

\begin{itemize}
\item 
\end{itemize}

\begin{itemize}
\item 
\end{itemize}

\begin{description}
\item 
\end{description}

\section{}

\end{document}
