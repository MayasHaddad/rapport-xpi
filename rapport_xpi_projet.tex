%% Based on a TeXnicCenter-Template by Tino Weinkauf.
%%%%%%%%%%%%%%%%%%%%%%%%%%%%%%%%%%%%%%%%%%%%%%%%%%%%%%%%%%%%%

%%%%%%%%%%%%%%%%%%%%%%%%%%%%%%%%%%%%%%%%%%%%%%%%%%%%%%%%%%%%%
%% HEADER
%%%%%%%%%%%%%%%%%%%%%%%%%%%%%%%%%%%%%%%%%%%%%%%%%%%%%%%%%%%%%
\documentclass[a4paper,10pt]{report}

\usepackage[utf8]{inputenc}
\usepackage[T1]{fontenc}
%% Language %%%%%%%%%%%%%%%%%%%%%%%%%%%%%%%%%%%%%%%%%%%%%%%%%
\usepackage[francais]{babel} %francais, polish, spanish, ...
\usepackage{lmodern} %Type1-font for non-english texts and characters

\usepackage{hyperref}

%% Packages for Graphics & Figures %%%%%%%%%%%%%%%%%%%%%%%%%%
\usepackage{graphicx} %%For loading graphic files


%%%%%%%%%%%%%%%%%%%%%%%%%%%%%%%%%%%%%%%%%%%%%%%%%%%%%%%%%%%%%
%% DOCUMENT
%%%%%%%%%%%%%%%%%%%%%%%%%%%%%%%%%%%%%%%%%%%%%%%%%%%%%%%%%%%%%
\begin{document}

\pagestyle{plain} %No headings for the first pages.


%% Title Page %%%%%%%%%%%%%%%%%%%%%%%%%%%%%%%%%%%%%%%%%%%%%%%
%% ==> Write your text here or include other files.
\pagenumbering{arabic}
%% The simple version:
\title{Projet d'XPI}
\author{Sébastien Bazin
\and Mayas Haddad
\and Ricky Keophila
\and Mohand-Saidh Hamitouche}
\maketitle
\tableofcontents

\chapter{Introduction}
Le présent document constitue le rapport accompagnant le rendu du projet d'XPI pour la promotion MIAGE de l'année 2013/2014.

\section{Problématique}
La problématique traitée par l'équipe de projet est celle de réaliser une plate-forme web d'achat et de location de films.
Voir le \href{https://www.lri.fr/~kn/files/projet_xpi_2013.pdf}{cahier des charges} pour plus d'informations.

\chapter{Choix de conception}

\section{Design Pattern MVC}

voir google pour le modele mvc.

\chapter{Choix d'implémentation}

\section{Twig pour les templates }
 
voir google.

\section{PHP Orienté objet pour la logique et le modèle}

l'inteeret dutiliser php orienté objet.

\section{MySQL }

parler de l'aspect que cela apporte niveaux sécurité.

\section{XML et XPath}

\chapter{Description des scénarios}

\section{Scénarios client}

\section{Scénario admin}

voir cahier des charges sauf vissionner 48h video et téléchargement des videos.
Sinon on list

\chapter{Répartition du travail}

\section{}

\subsection{}

\subsubsection{}

\paragraph{}

\subparagraph{}

\section{}

\begin{itemize}
\item 
\end{itemize}

\begin{itemize}
\item 
\end{itemize}

\begin{description}
\item 
\end{description}

\section{}

\end{document}
